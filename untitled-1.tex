\documentclass[12pt,a4paper]{article}
\usepackage{amssymb,amsmath,amsfonts,amsthm, nccmath, amscd}
\usepackage[utf8]{inputenc}
\usepackage[romanian,english]{babel}
\usepackage[T1]{fontenc} 
\usepackage{fullpage}
\usepackage{amsthm}
\usepackage{comment}
\usepackage{braket}
\usepackage{mmap}
\newtheorem{thm}{Theorem}
\newtheorem{lem}[thm]{Lemma}
\newtheorem{cor}[thm]{Corollary}
\newtheorem{rem}[thm]{Remark}
\newtheorem{remark}[thm]{Remark}
\newtheorem{conj}[thm]{Conjecture}
\newtheorem{defn}[thm]{Definition} 
\newtheorem{exmp}{Example}[section]
\newtheorem{prop}{Proposition}


\begin{document}
\begin{titlepage}
\begin{center}

\textsc{ \large UNIVERSITATEA BABEŞ-BOLYAI CLUJ-NAPOCA}\\\medskip

\textsc{\large FACULTATEA DE MATEMATICĂ ŞI INFORMATICĂ}\\\medskip
\textsc{\large SPECIALIZAREA MATEMATICĂ }\\[4 cm]

\textsc{\LARGE\textbf{GRADUATION THESIS} }\\[1 cm]
\textsc{\LARGE\textbf{Tensors}}\\[7 cm]
\begin{flushleft}
\textsc{\Large Conducător ştiinţific}\\
\textsc{\Large Prof. Dr. Andrei Mărcuş }\\[2 cm]
\end{flushleft}
\begin{flushright}
\textsc{\Large Absolvent}\\
\textsc{\Large Toth Imre}\\[3 cm]
\end{flushright}
\textsc{\Large Cluj Napoca\\ 2020}
\end{center}
\end{titlepage}
\renewcommand{\contentsname}{Cuprins}
\renewcommand{\refname}{Bibliografie}
\setcounter{page}{0}
\tableofcontents
\newpage
\section{Preliminaries}
\subsection{Vector space}
Before we dive into the notion of tensors, first we need to remember fundamental concepts that are necessary into understanding the tensors.

Let us begin with the definition of an abstract vector space.
\begin{defn} A \textbf{vector space} V over $\mathbb{C}$ is a set of objects called \textbf{vectors}, with the following properties:
\item[1.] To every pair of vectors x and y in V there corresponds a vector x + y also in V, called the \textit{sum} of x and y such that:
\item \hspace{1cm} a) x + y = y + x
\item \hspace{1cm} b) x + (y + z) = (x + y) + z
\item \hspace{1cm} c) There exists a unique vector 0 $\in$ V, called the \textbf{zero vector}, such that x + 0 = x for every vector x
\item \hspace{1cm} d) To every vector x $\in$ V there corresponds a unique vector -x $\in$ V such that x + (-x) = 0
\item[2.] To every complex number $\alpha$(also called a \textbf{scalar}) and every vector x there corresponds a vector $\alpha$x in V such that:
\item \hspace{1cm} a) $\alpha$($\beta$x) = ($\alpha$$\beta$)x
\item \hspace{1cm} b) 1x = x
\item[3.] Multiplication involving vectors and scalars is distributive:
\item \hspace{1cm} a) $\alpha$(x + y) = $\alpha$x + $\alpha$y
\item \hspace{1cm} b) ($\alpha$ + $\beta$)x = $\alpha$x + $\beta$x
\end{defn}
\begin{exmp}$\mathbb{C}^2$ is a vector space with multiplication and addition having their usual meaning.
\end{exmp}
Now we shall remember the inner product as it is defined, so we can revisit it later on when we talk about tensors.
\begin{remark}
The inner product generalizes the dot product to abstract vector spaces over a field of scalars, being either the field of real numbers $\mathbb{R}$ or the field of complex numbers $\mathbb{C}$. It is usually denoted by $\langle x, y\rangle$.
\end{remark}
\begin{defn}
The \textbf{inner product} of two vectors, x and y, in a vector space V is a complex number, $\langle x, y\rangle \in \mathbb{C}$ such that
\item[1.] $\langle x, y \rangle  = \langle y, x \rangle^*$
\item[2.] $\langle x, (\beta y + \gamma z) \rangle = \beta \langle x, y\rangle + \gamma \langle x, z\rangle$
\item[3.]$\langle x,x \rangle \geqslant 0 $ and $ \langle x, x \rangle = 0$ if and only if x = 0.\\
The last relation is called the \textbf{positive definite} property of the inner product. A positive definite real inner product is also called a \textbf{Euclidian} inner product, otherwise it is called \textbf{pseudo-Euclidian}.
\end{defn}
\begin{defn}
The vectors x$_{1}$,x$_{2}$,....,x$_{n}$ are said to be \textbf{linearly independent} if for $\alpha_{i} $$\in$$\mathbb{C}$, the relation $\sum_{i=1}^{n}$$ \alpha_{i}$x$_{i} $= 0 implies $\alpha_{i}$ =0 for all i. The  vectors are called \textbf{linearly dependent} otherwise. 
\end{defn}
\begin{defn}
A \textbf{subspace} W of a vector space V is a nonempty subset of V with the property that if x,y $\in$ W, then $\alpha$x + $\beta$y also belongs to W, $\forall$ $\alpha$,$\beta$ $\in$ $\mathbb{C}$
\end{defn}
\begin{remark}
The subspace is a vector space in its own right and the intersection of two subspaces is also a subspace.
\end{remark}
\begin{defn}
A \textbf{basis} of a vector space V is a set B of linearly independent vectors that spans all of V. A vector space that has a finite basis is called \textbf{finite-dimensional} and \textbf{infinite-dimensional} otherwise. We call the cardinality of the set B the \textbf{dimension} of V.
\end{defn}
\begin{defn}
A \textbf{linear map}(or \textbf{transformation}) from the complex vector space V to the complex vector space W is a mapping \textbf{T}:V$\to$W such that:\\
\hspace{1cm} \textbf{T}($\alpha$x + $\beta$y) = $\alpha$\textbf{T}(x) + $\beta$\textbf{T}(y) $\forall$ x,y$\in$ V and $\alpha$,$\beta$$\in$$\mathbb{C}$\\
A linear transformation \textbf{T}:V$\to$V is called an \textbf{endomorphism} of V or a \textbf{linear operator} on V.
\end{defn}
An important example of linear transformations occurs when the second vector space W, happens to be the set of scalars, $\mathbb{C}$ or $\mathbb{R}$, in which case the linear transformation is called a \textbf{linear functional}. The set of linear functionals $\mathcal{L}$(V,$\mathbb{C}$) or $\mathcal{L}$(V,$\mathbb{R}$) if V is a real vector space is denoted by V$^*$ and is called the \textbf{dual space} of V.
\begin{defn}
A vector space V is said to be isomorphic to another vector space W and written V ${\cong}$W, if there exists a bijective linear map \textbf{T}:
V $\to$ W. Then \textbf{T} is called an \textbf{isomorphism}. A bijective linear map of V onto itself is called an \textbf{automorphism} of V.An automorphism is also called an \textbf{invertible} linear map. 
\end{defn}
\begin{defn}
Let A be an N $\times$ N matrix. The mapping tr:M$^{(N\times N)}$$\to$$\mathbb{C}$ (or $\mathbb{R}$) given by tr A = $\sum_{i=1}^{n}$$ \alpha_{ii} $ is called the \textbf{trace} of A.
\end{defn}
\begin{thm}
The trace is a linear mapping. Furthermore, tr(AB) = tr(BA) and trA$^t$ = trA.
\end{thm}
\begin{proof}
To prove the first identity, we use the definitions of trace and matrix product:\\
tr(AB) = $\sum_{i=1}^{n}$(AB)$_{ii} $ = $\sum_{i=1}^{n}$ $\sum_{j=1}^{n}$(A)$_{ij}$(B)$_{ji}$ =  $\sum_{i=1}^{n}$ $\sum_{j=1}^{n}$(B)$_{ji}$(A)$_{ij}$ = $\sum_{j=1}^{n}$ ( $\sum_{i=1}^{n}$(B)$_{ji}$(A)$_{ij}$) = $\sum_{j=1}^{n}$(BA)$_{jj}$ = tr(BA).\\
The linearity of the trace and the second identity follow directly from the definition.
\end{proof}

\begin{thm} \label{inj0}
A linear transformation is 1–1 (injective) iff its kernel is
zero
\end{thm}
\begin{proof}
Suppose T |$a_1\rangle$ = T |$a_2\rangle$; \\\\Then linearity of T implies that T ($|a_1\rangle - |a_2\rangle)$ = 0. Since ker T = 0, we must have $|a_1\rangle = |a_2 \rangle$.
\end{proof}
\begin{thm} \label{inj}
A linear isometric map is injective.
\end{thm}
\begin{proof}
Let T : V $\to$ U be a linear isometry. Let |a$\rangle$ $\in$ kerT, then
\begin{center}
$\braket{a | a} = \braket{Ta | Ta} = \braket{0|0} = 0.$ 
\end{center}
Therefore, |a$\rangle$ = |0$\rangle$.\\
By Theorem $\ref{inj0}$, T is injective.
\end{proof}
\begin{thm} \label{dimthm}
Let T : V $\to$ W be a linear transformation. Then
\begin{center}
dim V = dim ker T + dim T(V)
\end{center}
This theorem is called the $\textbf{dimension theorem}$.
\end{thm}
\subsection{Multilinear maps}
There is a very useful generalization of the linear functionals that becomes essential in the treatment of tensors. However, a limited version of its application is used in the discussion of determinants, which we shall start here.

\begin{defn}
Let V and U be vector spaces. Let V$^P$ denote the p-fold Cartesian product of V. A \textbf{p-linear map} from V to U is a map $\Theta$:V$^P$$\to$U which is linear with respect to each of its arguments:\\
$\Theta$(x$_1$,...,$\alpha$x$_j$ + $\beta$y$_j$,...,x$_p$) = $\alpha$$\Theta$(x$_1$,....,x$_j$,....,x$_p$) + $\beta$$\Theta$(x$_1$,....,y$_j$,...,x$_p$).\\
A p-linear map from V to $\mathbb{C}$ or $\mathbb{R}$ is called a \textbf{p-linear function} in V.
\begin{exmp}
Let \{$\Phi$$_i$\}$^{p}_{i}$ be a linear functionals on V. Define $\Theta$ by $\Theta$(x$_1$,...,x$_p$) = $\Phi$$_1$(x$_1$),...,$\Phi$$_P$(x$_P$), x$_i$$\in$V.\\
Clearly $\Theta$ is p-linear.
\end{exmp}
\hspace {1cm} Let $\sigma$ denote the permutation of 1, 2,..., p. Define the p-linear map $\sigma$$\omega$ by $\sigma$$\omega$(x$_1$,...,x$_p$) = $\omega$(x$_{\sigma (1)}$,....,x$_{\sigma (p)}$)\\
\end{defn}
\begin{defn}
A p-linear map $\omega$ from V to U is \textbf{skew-symmetric} if $\sigma\omega$ = $\epsilon_{\sigma}$ if:\begin{center}
$\omega$(x$_{\sigma(1)}$,...,x$_{\sigma(p)}$) = $\epsilon_{\sigma}\omega$(x$_1$,...,x$_p$)\end{center}
where $\epsilon_{\sigma}$ is the sign of $\sigma$, which is +1 if $\sigma$ is even and -1 if it is odd. The set of p-linear skew-symmetric maps from V to U is denoted by $\Lambda^{P}$(V,U). The set of p-linear skew-symmetric functions in V is denoted by $\Lambda^{P}$(V).\\
\hspace{1cm} The permutation sign $\epsilon_{\omega}$ is sometimes written as \begin{center}
$\epsilon_{\sigma}$ = $\epsilon_{\sigma(1)\sigma(2)...\sigma(p)} $$\equiv$ $\epsilon_{i_{1}i_{2}...i_{p}}$, where i$_k$ $\equiv$ $\sigma(k)$.
\end{center}
\end{defn}


\begin{defn}
A skew symmetric N-linear function in V,i.e., a member of $\Lambda^N$(V) is called a \textbf{determinant function} in V.\\
Let B = \{|e$_k$\}$^{N}_{k=1}$ be a basis of V and B$^*$ = \{$\epsilon_k$\}$^N_{j=1}$ a basis of V$^*$, dual to B. For any set of N vector \{x$_k$\}$^N_{k=1}$ in V, define the N-linear function $\Theta$ by \begin{center}
	$\Theta$(x$_1$,...,x$_N$) = $\epsilon_1$(x$_1$)...$\epsilon_N$(x$_N$),
\end{center} 
Now let $\Delta$ be defined by $\Delta$ $\equiv$ $\sum_{\pi}\epsilon_{\pi} \cdot \theta$ Then, $\Delta \in \Lambda^N(V)$,i.e., $\Delta$ is a determinant function.
\end{defn}
Let \textbf{A} be a linear operator on an N-dimensional vector space V. Choose a nonzero determinant function $\Delta$. For a basis \{v$_i$\}$^{N}_{i=1}$ define the function $\Delta_A$ by \begin{center}
$\Delta_A$(v$_1$,...,v$_N$) $\equiv$ $\Delta$(Av$_1$,...,Av$_N$).
\end{center}
\begin{defn}
Let A $\in$ End(V). Let $\Delta$ be a nonzero determinant function in V, and let $\Delta_A$ be as mentioned before. Then \begin{center}
$\Delta_A$ = det A$\cdot \Delta$
\end{center}
defines the \textbf{determinant of A}.\end{defn}
\section{Tensors}
From here on,we will consider only real vector spaces and the basis vectors of a vector space V will be distinguished by a subscript and those of its dual space by a superscript.For example, if \{e$_i$\}$^N_{i=1}$ is a basis in V, then the basis in V$^*$ will be \{$\epsilon^j$\}$^N_{j=1}$ so we can avoid confusions.\textbf{ Einstein's summation convention} will also be used:\\
Repeated indices, of which one is an upper and the other a lower index, are assumed to be summed over: a$^k_i$b$^i_j$ means $\sum^N_{i=1}$a$^k_i$b$^i_j$.\\
It is more natural to label the elements of a matrix representation of an operator \textbf{A} by $\alpha^i_j$ (rather than $\alpha_{ji}$, because then Ae$_i$= $\alpha^j_i$e$_j$.
\subsection{Tensors as Multilinear Maps}
Since tensors are special kinds of linear operators on vector spaces, let us reconsider $\mathcal{L}$(V,W), the space of all linear mappings from the real vector space V to the real vector space W. 
\begin{defn}
A map \textbf{T}:V$_1\times$ V$_2\times $...$\times$ V$_r\to$ W is called r-\textbf{linear} if it is linear in all its variables:\begin{center}
\textbf{T}(v$_1$,....,$\alpha v_i$ + $\alpha'v'_i$,....,v$_r$) = $\alpha$\textbf{T}(v$_1$,....,v$_i$,....,v$_r$) + $\alpha'$\textbf{T}(v$_1$,....,v'$_i$,....,v$_r$) for all i.
\end{center}
\end{defn}
\begin{defn}
Let $\tau_1\in V^*_1$ and $\tau_2\in V^*_2$. We construct the bilinear map $\tau_1\otimes \tau_2:V_1\times V_2\to \mathbb{R}$ by $\tau_1\otimes \tau_2$(v$_1$,v$_2$) = $\tau_1$(v$_1$)$\tau_2$(v$_2$). The expression $\tau_1\otimes \tau_2$ is called the \textbf{tensor product} of $\tau_1$ and $\tau_2$
\end{defn}
An r-linear map can be multiplied by a scalar, and two r-linear maps can
be added; in each case the result is an r-linear map. Thus, the set of r-linear
maps from V$_1$ $\times$···$\times$ V$_r$ into W forms a vector space that is denoted by
$\mathcal{L}$(V$_1$,...,V$_r$;W).\\

We can also construct multilinear maps on the dual space. First, we note
that we can define a natural linear functional on V$^*$ as follows. We let $\tau \in V^* $
and v $\in$ V; then $\tau$(v) $\in$ $\mathbb{R}$. Now we twist this around and define a mapping
v : V$_*$ $\to$ $\mathbb{R}$ given by v($\tau$) $\equiv$ $\tau$(v). We have naturally constructed a linear functional on V$^*$ by
identifying (V$^*)^*$ with V.
\begin{defn}
Let V be a vector space with dual space V$^*$. Then a \textbf{tensor of type (r,s)} is a multilinear mapping\begin{center}
\textbf{T} $^r_s$:$\underbrace{V^*\times V^*\times ... \times V^*}_{r\ times}\times\underbrace{V\times V\times...\times V}_{s\ times}$
\end{center}
The set of all such mappings for fixed r and s forms a vector space denoted
by T$^r_s$ (V). The number r is called the \textbf{contravariant degree} of the tensor,
and s is called the \textbf{covariant degree} of the tensor.
\end{defn}
\begin{exmp}
\item[a)]A tensor of type (0, 0) is defined to be a scalar, so T$^0_0$(V) = $\mathbb(R)$.
\item[b]) A tensor of type (1, 0), an ordinary vector, is called a contravariant vector, and one
of type (0, 1), a dual vector (or a linear functional), is called a covariant
vector.
\item[c)] A tensor of type (r, 0) is called a contravariant tensor of rank r,
and one of type (0,s) is called a covariant tensor of rank s.
\end{exmp}
The union of T$^r_s$ (V) for all possible r and s can be made into an (infinite-dimensional) algebra called \textbf{algebra of tensors}.\\
First we define the following product on it:
\begin{defn}
The \textbf{tensor product} of a tensor \textbf{T} of type (r,s) and a tensor \textbf{U} of type (k,l) is a tensor \textbf{T}$\otimes$\textbf{U} of type (r+k,s+l),defined, as an operator on (V$^*$)$^{r+k}\times V^{s+l}$, by \begin{center}
\textbf{T}$\otimes$\textbf{U}($\theta^1$,...,$\theta^{r+k}$,u$_1$,...,u$_{s+l}$) = \textbf{T}($\theta^1$,...,$\theta^r$,u$_1$,....,u$_s$)\textbf{U}($\theta^{r+1}$,....,$\theta^{r+k}$,u$_{s+l}$,...,u$_{s+l}$).
\end{center}
This product turns the (infinite-dimensional) vector space of all tensors into an associative algebra called a \textbf{tensor algebra}.
\end{defn}
\begin{exmp}
What is the tensor product of \textbf{A} = 2e$_x$ - e$_y$ + 3e$_z$ with itself?\\
\textbf{A}$\otimes$\textbf{A} = (2,-1,3)$\otimes$(2-1,3)\\
Using the formula from above, we can compute the tensor product\\
(2,-1,3)$\otimes$(2-1,3)=(2 $\cdot$ 2,2 $\cdot$ -1,2 $\cdot$ 3,-1 $\cdot$ 2,-1 $\cdot$ -1,-1 $\cdot$ 3,3 $\cdot$ 2,3 $\cdot$ -1,3 $\cdot$ 3)\\
(2,-1,3)$\otimes$(2-1,3) = (4, -2, 6, -2, 1, -3, 6, -3, 9)

\end{exmp}
\textbf{Difference between direct sum and tensor product}\\
We'll assume V and W are finite dimensional vector spaces. That means we can think of V as $\mathbb{R}^n$ and W as $\mathbb{R}^m$ for some positive integers n and m.I will use the following simple example to show the difference between direct sum and tensor product:\\
Let x.y be two vectors, x $\in\mathbb{R}^3$ and y $\in\mathbb{R}^2$
\[x = 
\begin{pmatrix}
1\\2\\3
\end{pmatrix}
,y = 
\begin{pmatrix}
4\\5
\end{pmatrix}
\]
We call x $\oplus$ y the \textbf{direct sum} .In this case, x$\oplus$y is:
\[x = 
\begin{pmatrix}
1\\2\\3
\end{pmatrix}
\oplus
\begin{pmatrix}
4\\5
\end{pmatrix}
=\begin{pmatrix}
1\\2\\3\\4\\5
\end{pmatrix}
\]
We shall see that the direct sum give a list of m + n numbers, this gives us a way to build a space where the dimensions \textbf{add}.\\
Now lets see the tensor product of x and y, denoted x$\otimes$y.
\[x = 
\begin{pmatrix}
1\\2\\3
\end{pmatrix}
\oplus
\begin{pmatrix}
4\\5
\end{pmatrix}
=\begin{pmatrix}
1 \cdot 4 \\ 1 \cdot 5\\2 \cdot 4\\2 \cdot 5\\3 \cdot 4\\3 \cdot 5
\end{pmatrix}
\]
We see that the tensor product gives a list of m * n numbers, this gives us a way to build a space where the dimensions \textbf{multiply}.
\begin{defn}
A \textbf{contraction} of a tensor A$\in T^r_s$(V) with respect to a contravariant index at position p and covariant index at position q is a linear mapping C$^p_q$(V)$\to$T$^{r-1}_{s-1}$(V) given in \textit{component form} by\begin{center}
	[C$^p_q$(A)]$^{i_1...i_{r-1}}_{j_1...j_{s-1}}$ = A $^{i_1...i_{p-1}ki_{p+1}...i_r}_{j_1...j_{q-1}kj_{q+1}...j_s}$ $\equiv$ $\sum_{k}$A $^{i_1...i_{p-1}ki_{p+1}...i_r}_{j_1...j_{q-1}kj_{q+1}...j_s}$
\end{center}
\end{defn}
\subsection{Symmetries of Tensors}
Many applications demand tensors that have some kind of symmetry property. One symmetric tensor is the metric 'tensor' of an inner product: If V is a vector space  and v$_1$,v$_2$$\in$V, then g(v$_1$,v$_2$) = g(v$_2$,v$_1$). The following generalizes this property.
\begin{defn}
A tensor A is \textbf{symmetric} in the ith and jth variables if its value as a multilinear function is unchanged when these variables are interchanged.Clearly, the two variables must be of the same kind.
\end{defn}
\hspace{1cm} From this definition. it follows that in any basis, the components of a symmetric tensor do not change when the ith and jth indices are interchanged.
\begin{defn}
A tensor is \textbf{contravariant-symmetric} if it is symmetric in every pair of its covariant indices. A tensor is \textbf{symmetric} if it is both contravariant-symmetric and covariant-symmetric.
\end{defn}
\hspace{1cm} An immediate consequence of this definition is the following theorem:
\begin{thm}
A tensor S of type(r,0) is symmetric if and only if for any permutation $\pi$ of 1,2,...r, and any $\tau^1,\tau^2,...,\tau^r$ in V$^*$ we have \begin{center}
S($\tau^{\pi(1)},\tau^{\pi(2)},...,\tau^{\pi(r)}$) = S($\tau^1,\tau^2,...,\tau^n$)
\end{center}
\end{thm}
\begin{defn}
A \textbf{symmetrizer} is an operator S:T$_0^r\to$S$^r$ given by:\begin{center}
[S(A)]($\tau^1$,...,$\tau^r$) = $\dfrac{1}{r!}$$\sum_{\pi}$A($\tau^{\pi(1)},...\tau^{\pi(r)})$, where the sum is taken over the r! permutations of the integers 1, 2,...., r, and $\tau^1$,....$\tau^r$ are all in V$^*$. S(A) is often denoted by A$_s$.
\end{center}
\end{defn}
A$_s$ is a symmetric tensor. In fact,
A$_s$ ($\tau^{\sigma(1)},$...,$\tau^{\sigma(r)}$) =
[S(A)]($\tau^{\sigma(1)},$...,$\tau^{\sigma(r)}$)\\
\begin{flushright}
= $\dfrac{1}{r!}$$\sum_\pi$A($\tau^{\pi(\sigma(1))},...,\tau^{\pi(\sigma(r))})$\\
= $\dfrac{1}{r!}$$\sum_{\pi\sigma}$A($\tau^{\pi\sigma(1)},...,\tau^{\pi\sigma(r)})$\\
= A$_s(\tau^1,\tau^2,...\tau^r),$
\end{flushright}
where we have used the fact that the sum over $\pi$ is equal to the sum over the product (or composition) $\pi\sigma$, because they both include all permutations. Furthermore,if A is symmetric, then S(A) = A:\newline
[S(A)]($\tau^{1},....,\tau^{r}$) = $\dfrac{1}{r!}\sum_\pi$A($\tau^{\pi(1)}$,...,$\tau^{\pi(r)}$) = $\dfrac{1}{r!}\sum_\pi$A($\tau^1,...,\tau^r$) = $\dfrac{1}{r!}\underbrace{(\sum_\pi 1)}_{=r!}$A($\tau^1,...,\tau^r)$ = A($\tau^1,...,\tau^r$).\\
\hspace{1cm} A similar definition gives the symmetrizer S:T$_s^0\to$S$_s$. Instead of $\tau^1,...\tau^r$ in the definition, we would have v$_1$,...,v$_s$.
\begin{exmp}
For r = 2, we only have two permutations, and \begin{center}
A$_s$($\tau^1,\tau^2$) = $\dfrac{1}{2}$ [A($\tau^1,\tau^2$) + A($\tau^2,\tau^1$)].
\end{center}
For r = 3, we have six permutations 1, 2, 3, and the definition gives\begin{center}
A$_s$($\tau^1,\tau^2,\tau^3)$ = $\dfrac{1}{6}$[A($\tau^1,\tau^2,\tau^3$) + A($\tau^2,\tau^1,\tau^3$) + A($\tau^1,\tau^3,\tau^2$) + A($\tau^3,\tau^1,\tau^2$) + A($\tau^3,\tau^2,\tau^1$) + A($\tau^2,\tau^3,\tau^1$)].     
\end{center}
It is clear that the interchanging any pair of $\tau$'s on the right hand side of the above  two equations does not change the sum. Thus, A$_s$ is indeed a symmetric tensor.
\end{exmp}
\hspace{1cm} We are now ready to define a product on the collection of symmetric tensors and make it an algebra, called the \textbf{symmetric algebra}.
\begin{defn}
The \textbf{symmetric product} of symmetric tensors A$\in S^r$(V) and B$\in S^s$(V) is denoted by AB and defined as\\
AB($\tau^1$,...,$\tau^{r+s}$)$\equiv$ $\dfrac{(r+s)!}{r!s!}$S(A$\otimes$B)($\tau^1,...,\tau^{r+s}$)\\
= $\dfrac{1}{r!s!}\sum_{\pi}$A($\tau^{\pi(1)},...,\tau^{\pi(r)}$)B($\tau^{\pi(r+1)},...,\tau^{\pi(r+s)}$), where the sum is over all permutations of 1 2,...,r+s. The symmetric product of A$\in S_{r}$(V) and B$\in S_{s}$(V) is defined similarly.
\end{defn}
\begin{exmp}
Let us construct the symmetric tensor products of vectors. Fist we find the symmetric product of v$_1$ and v$_2$ both belonging to V = T$_0^1$(V):\\
(v$_1$,v$_2$)($\tau^1$,$\tau^2$) $\equiv$ v$_1$($\tau^1$)v$_2$($\tau^2$) + v$_1$($\tau^2$)v$_2$($\tau^1$)\\
=v$_1$($\tau^1$)v$_2$($\tau^2$) + v$_2$($\tau^1$)v$_1$($\tau^2$)\\
=(v$_1$$\otimes$ v$_2$ + v$_2$$\otimes$v$_1$)($\tau^1$,$\tau^2$).\\
Since it is true for any pair $\tau^1$ and $\tau^2$, we have \begin{center}
v$_1$v$_2$ = v$_1$$\otimes$v$_2$ + v$_2$$\otimes$v$_1$.
\end{center}
In general v$_1$,v$_2$...v$_r$ = $\sum_{\pi}$v$_{\pi(1)}\otimes v_{\pi(2)}\otimes ... \otimes v_{\pi(r)}.$\\
\hspace{1cm} It is clear from the definition that the symmetric multiplication is commutative, associative, and distributive. If we choose a basis \{e$_i$\}$_{i=1}^N$ for V and express all symmetric tensors in terms of symmetric products of e$_i$ using the above properties, then any symmetric tensor can be expressed as a linear combination of terms of the form $(e_1)^{n_1}$...($e_N)^{n_N}$.
\end{exmp}
\hspace{1cm} Skew-symmetry or \textbf{antisymmetry} is the same as symmetry except that in the interchange of variables the tensor changes sign.
\begin{defn}
A \textbf{covariant} (\textbf{contravariant}) skew-symmetric (or anti-symmetric) tensor is one that is skew-symmetric in all pairs of covariant (contravariant) variables. A tensor is skew-symmetric if it is both covariant
and contravariant skew-symmetric.
\end{defn}
\begin{thm}
A tensor A of type(r, 0) is skew if and only if for any permutations $\pi$ of 1, 2, ... r, and any$\tau^1,\tau^2, ..., \tau^r$ in V$^*$. we have\begin{center}
$A(\tau^{\pi(1)},\tau^{\pi(2)},....,\tau^{\pi(r)}$ = $\epsilon_\pi$ A($\tau^1,\tau^2,...,\tau^r$).
\end{center}
\end{thm}
\begin{defn}
An \textbf{antisymmetrizer} is a linear operator A on T$^r_0$, given by \begin{center}
[A(T)]($\tau^1,...,\tau^r$) = $\dfrac{1}{r!}\sum_{\pi}$T($\tau^{\pi(1)},....,\tau^{\pi(r)}$.
\end{center}
\end{defn}
A(T) is denoted by T$_a$.\\
\hspace{1cm} Clearly, T$_a$ is an antisymmetric tensor. In fact, using ($\epsilon_\sigma$)$^2$ = 1, which holds for any permutation, we have\\
T$_a \tau^{\sigma(1)},....,\tau^{\sigma(r)}$ = [A(T)]($\tau^{\sigma(1)},...,\tau^{\sigma(r)}$)\\
=$(\epsilon_\sigma)^2 \dfrac{1}{r!} \sum_{\pi} \epsilon_\pi$A($\tau^{\pi\sigma(1)},...,\tau^{\pi\sigma(r)}$)\\
=$\epsilon_\sigma \dfrac{1}{r!} \sum_{\pi\sigma} \epsilon_\pi \epsilon_\sigma$T($\tau^{\pi\sigma(1)},...,\tau^{\pi\sigma(r)}$)\\
=$\epsilon_\sigma T_a$($\tau^1,\tau^2,....\tau^r$).\\
\hspace{1cm} where we used the fact that $\epsilon_\pi \epsilon_\sigma$ = $\epsilon_{\pi\sigma}$. If T is antisymmetric, then A(T) = T:\newline
[A(T)]($\tau^1,...,\tau^r)$ = $\dfrac{1}{r!}\sum_{\pi}\epsilon_\pi$T($\tau^{\pi(1)},...,\tau^{\pi(r)}$)\\
=$\dfrac{1}{r!}\sum_{\pi}(\epsilon_\pi)^2 T(\tau^1,...\tau^r)$\\
=$\dfrac{1}{r!} \sum_{\pi} 1$ T($\tau^1,...,\tau^r) = T(\tau^1,...\tau^r).$\newline
\hspace{1cm} A similar definition gives the antisymmetrizer A on T$_s^0$. Instead of $\tau^1,...\tau^r$ we would have used $v_1,...v_s$.

\begin{exmp}
Let us write the equation of an antisymmetrizer for r = 3.\\
T$_a(\tau^1,\tau^2,\tau^3)$ = $\dfrac{1}{6} [\epsilon_{123}$A($\tau^1,\tau^2,\tau^3)$ +$ \epsilon_{213}$A($\tau^2,\tau^1,\tau^3) $+ $\epsilon_{132}$A($\tau^1,\tau^3,\tau^2)$ +$ \epsilon_{312}$A($\tau^3,\tau^1,\tau^2)$ + $\epsilon_{321}$A($\tau^3,\tau^2,\tau^1)$ + $\epsilon_{231}$A($\tau^2,\tau^3,\tau^1)$ ]\\
=$\dfrac{1}{6}[$A($\tau^1,\tau^2,\tau^3)$- A($\tau^2,\tau^1,\tau^3) $ - A($\tau^1,\tau^3,\tau^2)$ + A($\tau^3,\tau^1,\tau^2)$ - A($\tau^3,\tau^2,\tau^1)$ + A($\tau^2,\tau^3,\tau^1)$ ].
\end{exmp}

\subsection{Exterior algebra}
The following discussion will concentrate on tensors of type (r, 0). However, interchanging the roles of V and V$^*$ makes all definitions, theorems,
propositions, and conclusions valid for tensors of type (0,s) as well. The set of all skew-symmetric tensors of type (p, 0) forms a subspace of T$^{p}_{0}$. This subspace is denoted by $\Lambda^{p}  (V^*)$ and its members are called \textbf{p-vectors} It is not, however, an algebra unless we define a skew-symmetric
product analogous to that for the symmetric case. This is done in the following definition:
\begin{defn}
The \textbf{exterior product} (also called the wedge, Grassmann, alternating, or veck product) of two skew-symmetric tensors \textbf{A}$\in\Lambda^{p}(V^*)$ and  \textbf{B}$\in\Lambda^{q}(V^*)$ is a skew-symmetric tensor belonging to $\Lambda^{p+q}(V^*)$ and given by
\begin{center}
A $\wedge$ B $\equiv$ $\dfrac{(r+s)!}{r!s!}$ $\mathbb{A}$ (A $\otimes$ B ) =$\dfrac{(r+s)!}{r!s!}$(A $\otimes$ B ) $_{a}$.
\end{center}
\end{defn}

More explicitly,\\
A $\wedge$ B = ($\tau^1$,...,$\tau^{r+s}$)\\	
= $\dfrac{1}{r!s!}\sum_{\pi}\epsilon_{\pi} A (\tau^{\pi(1)},...,\tau^{\pi(r)})B (\tau^{\pi(r+1)},...,\tau^{\pi(r+s)})$

%add example 
\begin{thm}
The exterior product is associative and distributive
with respect to the addition of tensors. Furthermore, it satisfies the
following anticommutativity property:
\begin{center}
A $\wedge$ B = (-1)$^{pq}B\wedge A$
\end{center}
whenever A $\in$ $\Lambda^{p} (V^*)$ and B $\in$ $\Lambda^{q} (V^*)$ .  In particular, v$_1$ $\wedge$ v$_2$ =
   - v$_2$ $\wedge$ v$_1$ for v$_1$, v$_2$ $\in$ V
\end{thm}
\begin{defn}
The elements of $\Lambda^{p} (V^*)$ are called \textbf{p-forms} 
\end{defn}
A linear transformation T : V $\to$ W induces a transformation T$^*$ : $\Lambda^{p} (W)$  $\to$ $\Lambda^{p} (V)$ defined by
\begin{center}
($T^{*} \rho (v_1,...,v_p) \equiv \rho (T v_1,...,T v_p)$ , $\rho \in \Lambda^{p}(W)$, $v_i \in V$
\end{center}

T$^*\rho$ is is called the \textbf{pullback } of $\rho$ by T. The most important properties of
pullback maps are given in the following:\\
Let T : V $\to$ W and  S : W $\to$ U. Then\\
\\
1. T$^* : \Lambda^{p} (W) \to \Lambda^{p} (V)$\\
2. (S $\circ$ T)$^*$ = $T^* \circ S^*$\\
3. If T is the identity map, so is $T^*$\\
4. If T is an isomorphism, so is $T^*$ and $(T^*)^{-1} = (T^{-1})^*$\\
5. If $\rho \in \Lambda^{p} (W)$ and $\sigma \in \Lambda^{p} (W)$, then T$^*$( $\rho \wedge \sigma $) = $T^* \rho \wedge T^* \sigma$.
\\ \\
If \{e$_i$\}$^N_{i=1}$ is a basis of V, we can form a basis for $\Lambda^{p}(V^*)$ by constructing all products of the form $e_{i_1} \wedge e_{i_2} \wedge ... \wedge e_{1_p}$.  The number of linearly independent such vectors, which is the dimension of  $\Lambda^{p}(V^*)$,  , is equal to the number
of ways p numbers can be chosen from among N distinct numbers in such
a way that no two of them are equal. This is simply the combination of N
objects taken p at a time. Thus, we have \\
\begin{center}
 dim ($\Lambda^{p}(V^*)$) =  $\begin{pmatrix} N \\ p \end{pmatrix}$ =  $\dfrac{N!}{p!(N - p)!}$
\end{center}

In particular dim ($\Lambda^{p}(V^*)$) = 1.\\
\indent Any A $\in \Lambda^{p}(V^*)$ can be written as 
\begin{center}
A = $\sum^{N}_{i_1 < i_2 < ... < i_p}$ A$^{i_1...i_p}$  $e_{i_1} \wedge e_{i_2} \wedge ... \wedge e_{1_p}$ \\
= $\dfrac{1}{p!} \sum^{N}_{i_1,i_2 ... i_p}$ A$^{i_1...i_p}$  $e_{i_1} \wedge e_{i_2} \wedge ... \wedge e_{1_p}$
\end{center}
where A$^{i_1...i_p}$ are the components of A, which are assumed completely antisymmetric in all $i_1,i_2 ... i_p$. In the second sum, all i’s run from 1 to N.\\
\begin{thm}
Set $\Lambda^{0}(V^*)$ = $\mathbb{R}$ and let $\Lambda(V^*)$ denote the direct
sum of all  $\Lambda^{p}(V^*)$:
\begin{center}
$\Lambda(V^*)$  = $\oplus^{N}_{p = 0}$  $\Lambda(V^*)$ $\equiv$ $\mathbb{R} \oplus V \oplus \Lambda^{2}(V^*) \oplus \Lambda^{N}(V^*)$
\end{center}
Then $\Lambda(V^*)$ is a $2^N$-dimensional algebra with exterior product defining its multiplication rule.	
\end{thm}

\indent Given two vector spaces V and U, one can construct a tensor product
of $\Lambda(V^*)$ and $\Lambda(U^*)$ and define a product $\odot$ on it as follows. Let $A_i \in \Lambda^{p_i}(V^*)$, i = 1, 2 and  $B_j \in \Lambda^{q_j}(U^*)$, j = 1, 2. Then \\
\begin{center}
$(A_1 \otimes B_1) \odot (A_2 \otimes B_2) \equiv (-1)^{p_2 q_1}(A_1 \wedge A_2) \otimes ( B_1 \wedge B_2). $
\end{center}
\begin{defn}
The tensor product of the two vector spaces $\Lambda(V^*)$ and $\Lambda(U^*)$ together with the product given before is called $\textbf{skew tensor product}$ of  $\Lambda(V^*)$ and $\Lambda(U^*)$ and denoted by  $\Lambda(V^*) \overset{\wedge}{\otimes} \Lambda(U^*)$
\end{defn}
An elegant way of determining the linear independence of vectors using
the formalism developed so far is given in the following proposition.\\
\begin{prop}\label{prop1}
A set of vectors v$_1,...,v_p$ is linearly independent if
and only if $v_1 \wedge ... \wedge v_p$ $\ne$ 0.
\end{prop}
\begin{proof}
If \{v$_i$\}$^p_{i=1}$ are independent, then they span a p-dimensional subspace M of V. Considering M as a vector space in its own right, we have
dim  $\Lambda^{p}(M^*)$ = 1. A basis for $\Lambda^{p}(M^*)$ is simply $v_1 \wedge ... \wedge v_p$ , which cannot
be zero. \\
\indent Conversely, suppose that $\alpha_1 v_1 + ... \alpha_p v_p $= 0. Then taking the exterior
product of the left hand side with $v_2 \wedge v_3 \wedge ... \wedge v_p$ makes all terms vanish (because
each will have two factors of a vector in the wedge product) except the first
one. Thus, we have $\alpha_1 v_1 \wedge ... \wedge v_p $.  The fact that the wedge product is
not zero forces $\alpha_1$ to be zero. Similarly, multiplying by $v_1 \wedge v_3 \wedge ... \wedge v_p$ shows that $\alpha_2$ = 0, and so on.
\end{proof}
\begin{exmp}
Let \{e$_i$\}$^N_{i=1}$ be a basis for V. Let $v_1 = e_1 + 2e_2 - e_3$ $v_2 = 3e_1 + e_2 + 2e_3$ , $v_3 = -e_1 - 3e_2 + 2 e_3$.
\begin{center}
$v_1 \wedge v_2$ = ($ e_1 + 2e_2 - e_3$) $\wedge$ ($3e_1 + e_2 + 2e_3$)\\
= -5$e_1 \wedge e_2$ + 5$e_1 \wedge e_3$ + 5 $e_2 \wedge e_3$.
\end{center}	
All the wedge products that have repeated factors vanish. Now we multiply
by $v_3$: 
\begin{center}
$v_1 \wedge v_2 \wedge v_3$ = -5$e_1 \wedge e_2 \wedge ( -e_1 - 3e_2 + 2e_3) $\\$+ 5e_1 \wedge e_3 \wedge ( -e_1 -3e_2 + 2e_3)  $\\$+5e_2 \wedge
e_3 \wedge (-e_1 - 3e_2 + 2e_3)$\\
= $-10e_1 \wedge e_2 \wedge e_3 - 15e_1 \wedge e_3 \wedge e_2 - 5e_2 \wedge e_3 \wedge e_1$ = 0.
\end{center}
We conclude that the three vectors are linearly dependent.
\end{exmp}
As an aplication of Proposition 1 let us prove the following.
\begin{thm}
(Cartan’s lemma) Suppose that \{e$_i$\}$^p_{i=1}$ , p $\leq$ dim V form
a linearly independent set of vectors in V and that  \{v$_i$\}$^p_{i=1}$ are also vectors
in V such that $\sum^{p}_{i=1} e_i \wedge v_i$ = 0. Then all $v_i$ are linear combinations of only
the set \{e$_i$\}$^p_{i=1}$. Furthermore, if $v_i$ = $\sum^p_{j=1} A_{ij}e_{j}$, then $A_{ij} = A_{ji}$.
\end{thm}
\begin{proof}
Multiplying both sides of $\sum^{p}_{i=1} e_i \wedge v_i$ = 0 by $e_2 \wedge ... e_p$ gives
\begin{center}
$-v_1 \wedge e_1 \wedge e_2 \wedge ... \wedge e_p = 0.$
\end{center}
By Proposition 1, $v_1$ and the $e_i$ are linearly dependent. Similarly, by
multiplying $\sum^{p}_{i=1} e_i \wedge v_i$ = 0 by the wedge product with $e_k$  we
show that $v_k$ and  the $e_i$ are linearly dependent. Thus, $v_k = \sum^p_{i=1} A_{ki} e_i$, for all k. Furthermore, we have\\
\begin{center}
 0 = $\sum^{p}_{k=1} e_k \wedge v_k$  = $\sum^{p}_{k=1} \sum^{p}_{i=1} e_k \wedge ( A_{ki} e_i) = \sum_{k<i} (A_{ki} - A_{ik}) e_k \wedge e_i$
\end{center}
where the last sum is over both k and i with k<i. Clearly, \{$e_k \wedge e_i$\} with k <
i are linearly independent . Therefore, their coefficients must
vanish.
\end{proof}
\begin{defn}
The symbol $\epsilon_{i_1, i_2, ... , i_n}$ called the $\textbf{Levi-Civita tensor}$, can be defined by
\begin{center}
$\epsilon^{i_1} \wedge ... \wedge \epsilon^{i_N} $ = $\epsilon_{i_1 ... i_N} \epsilon^1 \wedge ... \epsilon^N$.
\end{center}
\end{defn}
\begin{prop}
The Levi-Civita tensor $\epsilon_{i_1, i_2, ... , i_n}$ takes the same value in
all coordinate systems.
\end{prop}
\begin{defn}
A U-valued p-form, is a linear machine that takes p
vectors from V and produces a vector in U. The space of U-valued p-forms
is denoted by $\Lambda^{p}(V, U)$.  In this new context $\Lambda^{p}(V)$ =  $\Lambda^{p}(V, \mathbb{R})$.
\end{defn}

\subsection{Orientation}
\begin{defn}
An $\textbf{oriented basis}$ of an N-dimensional vector space is
an ordered collection of N linearly independent vectors.
\begin{center}
If \{v$_i$\}$^N_{i=1}$ is one oriented basis and \{u$_i$\}$^N_{i=1}$ is a second one, then\\
$u_1 \wedge u_2 \wedge ... \wedge u_N = (det R)v_1 \wedge v_2 \wedge ... \wedge v_n,$
\end{center}
where R is the transformation matrix and det R is a nonzero number (R is
invertible), which can be positive or negative
\end{defn}
 \indent Accordingly, we have the following definition
\begin{defn}
An $\textbf{orientation}$ is the collection of all oriented bases related by a transformation matrix having a positive determinant. A vector
space for which an orientation is specified is called an $\textbf{oriented}$ vector space.
\end{defn}
\indent Clearly, there are only two orientations in any vector space. Each oriented
basis is positively related to any oriented basis belonging to the same orientation and negatively related to any oriented basis belonging to the other
orientation. For example, in $\mathbb{R}^3$,  the bases \{$e_x, e_y, e_z$\} and \{$e_y, e_x, e_z$\} belong
to different orientations because
\begin{center}
$e_x \wedge e_y \wedge e_z$ = -$e_y \wedge e_x \wedge e_z$.
\end{center}

The first basis is (by convention) called a right-handed coordinate system,
and the second is called a left-handed coordinate system. Any other basis
is either right-handed or left-handed. There is no third alternative!

\begin{defn}
Let V be a vector space. Let V$^{*}$ have the oriented basis \{$\epsilon_i$\}$^N_{i=1}$. The oriented $\textbf{volume element}$ $\mu \in \Lambda^{N}(V)$ of V is defined as
\begin{center}
$\mu \equiv \epsilon^1 \wedge \epsilon^2 \wedge ... \wedge \epsilon^N.$
\end{center}
\end{defn}
\indent Note that if \{$e_i$\} is as ordered as \{$\epsilon^j$\} then $\mu( e_1, e_2, ... , e_N) = +1/N!$ and we say that \{$e_i$\} is positively oriented with respect to $\mu$. In general, \{$v_i$\}  is
positively oriented with respect to $\mu$ if $\mu(v_1, v_2, ... , v_N) > 0$. \\ The volume element of V is defined in terms of a basis for V$^*$
\subsection{Symplectic Vector Spaces}

\begin{defn}
A 2-form $\omega \in \Lambda^{2}(V)$ is $\textbf{nondegenerate}$ if $\omega(v_1, v_2) = 0$ for all $v_1 \in V$ implies $v_2 = 0$. \\A $\textbf{symplectic form}$ on V is a nondegenerate
2-form $\omega \in \Lambda^{2}(V)$. \\The pair(V,$\omega$) is called a $\textbf{symplectic vector space}$.\\
If (V,$\omega$) and (W,$\rho$) are symplectic vector spaces, a linear transformation $T:V \to W$ is called a $\textbf{symplectic transformation}$ or a $\textbf{symplectic map}$ if T$^* \rho = \omega$.
\end{defn}

Any 2-form (degenerate or nondegenerate) leads to other quantities that
are also of interest. For instance, given any basis \{$v_i$\} in V, one defines the
$\textbf{matrix}$ of the 2-form $\omega \in \Lambda^{2}(V)$ by $\omega_{ij} \equiv \omega$(v$_i$, v$_j$ ). Similarly, one can
define the useful linear map $\omega^{\flat}$ : V$\to$ V$^*$ by
\begin{center}
$ [ \omega^{\flat} (v) ] v^{'} \equiv \omega(v, v^{'})$.
\end{center}
\indent where $ [ \omega^{\flat} (v) ] v^{'} \in \mathbb{R}$ and  $ [ \omega^{\flat} (v) ] \in V^{*}$\\
The rank of $\omega^{\flat}$ is called the $\textbf{rank}$ of $\omega$.
\begin{rem}
A 2-form $\omega$ is nondegenerate if and only if the determinant of ($\omega_{ij}$ ) is nonzero, if and only if $\omega^{\flat}$ is an isomorphism, in which
case the inverse of $\omega^{\flat}$ is denoted by $\omega^{\sharp}$.
\end{rem}
%maybe add more stuff idk
\begin{prop}
Let (V,$\omega$) be a symplectic vector space. Then the set of
symplectic mappings T : (V,$\omega$) $\to$ (V,$\omega$) forms a group under composition,
called the $\textbf{symplectic group}$ and denoted by Sp(V,$\omega$).
\end{prop}
\subsection{Inner Product}
\begin{defn}
A $\textbf{symmetric bilinear form}$ b on V is a symmetric tensor of type (0, 2).
\end{defn}
\indent If \{e$_j$\}$^N_{j=1}$ is a basis of V and \{$\epsilon^i$\}$^N_{i=1}$ is its dual basis, then b = $\dfrac{1}{2}b_{ij}\epsilon^i\epsilon^j$, because $\epsilon^i \epsilon^j $ = $\epsilon^i \otimes \epsilon^j$ + $\epsilon^j \otimes \epsilon^i$ form a basis of S$_2$(V). If v and u are any two vectors in V, then
\begin{center} \label{prop2}
b(v,u) = $\dfrac{1}{2}b_{ij}(\epsilon^i \otimes \epsilon^j$ + $\epsilon^j \otimes \epsilon^i)$($v^k e_k , u^m e_m)$\\
=$\dfrac{1}{2}b_{ij}v^k u^m [ \epsilon^i (e_k) \epsilon^j (e_m) + \epsilon^j (e_k) + \epsilon^i (e_m)]$\\
=$\dfrac{1}{2}b_{ij}v^k u^m [ \zeta_k^i \zeta_m^j + \zeta_k^j \zeta_m^i]$\\
=$ \dfrac{1}{2}b_{ij} ( v^i u ^j + v^j u^i)$ \\
=$ b_{ij} v^i u^j$\\
\end{center}
For any vector v $\in$ V, we can write
\begin{center}
b(v) = $\dfrac{1}{2}b_{ij} \epsilon^i \epsilon^j (v)$ \\
= $\dfrac{1}{2}b_{ij}(\epsilon^i \otimes \epsilon^j$ + $\epsilon^j \otimes \epsilon^i)$($v^k e_k )$\\
=$\dfrac{1}{2}b_{ij}v^k [ \epsilon^i \epsilon^j (e_k) \epsilon^j  \epsilon^i (e_k) ]$\\
=$\dfrac{1}{2}b_{ij}v^k [ \epsilon^i \zeta_k^j (e_k) \epsilon^j  \zeta^i_k (e_k) ]$\\
=$ \dfrac{1}{2}b_{ij} [ v^j \epsilon ^i + v^i \epsilon^j]$ \\
=$b_{ij}v^j\epsilon^i$\\
=$b_{ij}v^i\epsilon^j$.
\end{center}

Thus, b(v)$\in$ V$^*$. This shows that b can be thought of as a mapping from
V to V$^*$, which we denote by $b_*$ and write $b_*$ : V $\to$ V$^*$. For this mapping
to make sense, it should not matter which factor in the symmetric product v
contracts with. But this is a trivial consequence of the symmetries b$_{ij}$ = b$_{ji}$
and $\epsilon^i \epsilon^j = \epsilon^j \epsilon^i.$\\
\\Let v and u be any two vectors in V. Let  \{e$_j$\}$^N_{j=1}$ be a basis of V and \{$\epsilon^i$\}$^N_{i=1}$ its dual basis. The natural pairing of v and b$_*$(u) is given by
\begin{center}
$\braket{b_*(u), v} = $\\
=$\braket{b_{ij} u^j \epsilon^i, v^k e_k} $\\
= $b_{ij}u^j v^k \braket{\epsilon^i, e_k}$\\
=$b_{ij}u^j v^k \zeta^i_k$\\
=$b_{ij}u^j v^i$\\
=$b(u,v)$\\
=$b(v,u)$

\end{center}
where we used Proposition $\ref{prop2}$ in the last step.\\
\\
The components $b_{ij}v^{j}$ of $b_*$(v) in the basis of \{$\epsilon^i$\}$^N_{i=1}$ of $V^*$ are denoted by v$_i$, so
\begin{exmp} \label{eq1}
\begin{center}
$b_*(v) = v_i \epsilon^i$
\end{center}
\end{exmp}
where $v_i = b_{ij} v^j$.\\
\\
We have thus $\textbf{lowered}$ the index of v$^j$ by the use of the symmetric bilinear
form b. In applications v$_i$ is uniquely defined; furthermore, there is a one-to-one correspondence between v$_i$ and v$^i$
. This can happen if and only if the
mapping b$_*$ : V $\to$ V$^*$ is invertible, in which case b is usually denoted by g.
If g$_*$ is invertible, there must exist a unique $(g_*)^-1 \equiv (g^-1)_* : V^* \to V$, or $g^-1 \in S_2(V^*)= S_2(V)$, such that
\begin{center}

$v^j e_j$ = v = \\
= $(g_*)^{-1} g_*(v)$\\
=$(g_*)^{-1} (v_i \epsilon^i)$\\
= $v_i (g_*)^{-1}(\epsilon^i)$\\
=$v_i [ (g^{-1})^{jk} e_j e_k ]( \epsilon^i) $\\
=$v_i (g^{-1}) ^ {jk} e_j \underbrace{e_k (\epsilon^i)}_{= \zeta^i_k}$\\
=$v_i (g^{-1})^{ji} e_j$.
\end{center}
\indent Comparison of the LHS and the RHS yield $v^j$ = $v_i(g^{-1})^{ji}$. It is customary
to omit the -1 and simply write \\
\begin{exmp}\label{eq2}
\begin{center} 
$v^j = g ^ {ji} v_{i}$,
\end{center}
\end{exmp}
where it is understood that g with upper indices is the inverse of g (with
lower indices). \\
\begin{defn}
An invertible bilinear form is called $\textbf{nondegenerate}$.
\end{defn}
\begin{defn}
A symmetric bilinear form g that is nondegenerate is called an $\textbf{inner product}$. When there is no danger of confusion, we write $\braket{u, v}$ instead of g(u, v).
\end{defn}
% TODO add deffinitions
We therefore see that the presence of a nondegenerate symmetric bilinear
form (or an inner product) naturally connects the vectors in V and V$^*$in a
unique way. For any vector v$\in$V there is a unique linear functional $\phi_u \in V^*$
given by $\phi_u = g_*(v)$. An inner product
makes a vector space self-dual. \\There exists a determinant function $\Delta_0$ such that
\begin{center}
$\Delta_0 (v_1, ... , v_N) \Delta_0 ( u_1, ..., u_N) = \alpha det(g(v_i, u_j))$.
\end{center}
This is called the $\textbf{Lagrange identity}$\\
\\
Going from a vector in V to its unique image in $V^*$ is done by simply $\textbf{lowering the index}$ using Eq. $\ref{eq1}$, and going the other way involves using Eq. $\ref{eq2}$ to $\textbf{raise the index}$.\\
\\
This process can be generalized to all tensors. For instance, although in general, there is no connection among T$^2_0(V), T^1_1(V),$ and $T_2^0$(V), the introduction of an inner product connects all
these spaces in a natural way and establishes a one-to-one correspondence
among them. Thus, to a tensor in T$^2_0(V)$ with components $t^{ij}$ j there corresponds a unique tensor in $T^1_1(V)$ given, in component form, by $t_j^i$ = $g_{jk}t^{ik}$, and another unique tensor in $T_2^0$(V), given by $t_{ij} = g_{il} t^l_j = g_{il} g_{jk} t^{lk}$.\\\\
Let us apply this technique to g$^{ij}$ , which is also a tensor and for which
the lowering process is defined. We have
\begin{center}
$g^i_j = g_{ik}g^{kj} = (g^{-1})^{ik} g_{kj} = \zeta_j^i$.
\end{center}
This relation holds in all bases.\\
\\
The inner product has been defined as a nondegenerate symmetric bilinear form. The important criterion of nondegeneracy has equivalences:
\begin{prop} \label{propdeg}
A symmetric bilinear form g is nondegenerate if
and only if\\ \\
1. the matrix of components $g_{ij}$ has a nonvanishing determinant or,\\ \\
2. for every nonzero v $\in$ V, there exists w $\in$ V such that g(v,w) $\ne$ 0.
\end{prop}
\begin{proof}
The first part is a direct consequence of the definition of nondegeneracy. The second part follows from the fact that $g_*$ : V $\to$ V$^*$ is invertible if and only if the nullity of $g_*$ is zero.  It follows that if $v \in V$ is nonzero, then $g_*(v) \ne 0,$ i.e., $g_*(v)$ is not the zero functional. Thus, there must exist a
vector w $\in$ V such that [ $g_*$(v)](w) $\ne$ 0. The proposition is proved once we
note that [$g_*$(v)](w) $\equiv$ g(v,w).
\end{proof}
Let (V,g) and (U,h) be inner product spaces. Recall that an isometry is
a linear transformation T : V $\to$ U which preserves the inner product, i.e.,
\begin{center}
$g(v_1, v_2) = h(T v_1, T v_2).$
\end{center}
It was shown in Theorem $\ref{inj}$ that  an isometry is injective. However, the
proof relied on the positive definiteness of the inner product. That is not
necessary. In fact, suppose that Tv = 0. Then, for any x $\in$ V, we have
\begin{center}
g(x, v) = h(Tx,Tv) = h(Tx,0) = 0.
\end{center}
By proposition $\ref{propdeg}$ since g is nondegenerate. It follows that
ker T = \{0\}, and we have
\begin{thm}
A linear isometry is injective for all inner products.
\end{thm}
\begin{defn}
The $\textbf{g-transpose}$ of a linear endomorphism T : V $\to$ V is
the endomorphism T$^t$ given by
\begin{center}
g($T^t u, v)$ = g(u, Tv).
\end{center}
\end{defn}

If T is an is isometry, then
\begin{center}
g(u, v) = g(Tu, Tv) = g(T^t Tu, v).
\end{center}
Since this holds for arbitrary u and v, we must have T$^t$
T = 1. Thus,
\begin{prop}
An endomorphism T : V$\to$V is an isometry if and only
if T$^t$ = T$^{−1}$.
\end{prop}
\begin{defn} 
A symmetric bilinear form b can be categorized as follows:
\\\\
1. positive (negative) definite: b(v, v) > 0 [b(v, v) < 0] for every
nonzero vector v;\\
2. definite: b is either positive definite or negative definite;
3. positive (negative) semidefinite: b(v, v) ≥ 0 [b(v, v) ≤ 0] for every v;
4. semidefinite: b is either positive semidefinite or negative semidefinite;
5. indefinite: b is not definite.
\end{defn}
If b is a symmetric bilinear form on V, then the restriction b|$_W$ of b on a
subspace W is also symmetric and bilinear, and if b is definite or semidefinite, then so is b|$_W$.
\begin{defn}
The $\textbf{index}$ ν of a symmetric bilinear form b on V is the
dimension of the largest subspace W of V on which b|$_W$ is negative definite.\\\\
Sometimes ν is referred to as the index of V.
\end{defn}
\begin{exmp}
Some of the categories of the definition above can be illustrated in $\mathbb{R}^2$ with $v_1 = (x_1, y_1)$, $v_2 = (x_2, y_2)$, and v = (x, y).\\\\
1. Positive definite: b$(v_1, v_2)$ = $x_1 x_2 + y_1 y_2$ because if v $\ne$ 0, then one of its components is nonzero and b(v, v) = $x^2$ + $y^2$ > 9. \\
2. Negative definite: b($v_1, v_2)$ = $\dfrac{1}{2}(x_1 y_2 + x_2 y_1) - x_1 x_2 - y_1 y_2 because 
\begin{center}
b(v, v) = x y - $x^2 - y^2$ = - $\dfrac{1}{2}(x - y)^2$ - $\dfrac{1}{2}x^2 -\dfrac{1}{2} y^2,
\end{center}
which is negative for nonzero v.
3 . Indefinite: b($v_1,v_2$) = $x_1 x_2 - y_1 y_2$. For v = (x, x), b(v, v) = 0. However, b is nondegenerate, because it has the invertible matrix g = \begin{pmatrix}
1 & 0 \\
0 & -1 
\end{pmatrix} in the standard basis for $\mathbb{R}^2$.
4. Positive semidefinite: b($v_1, v_2$) = $x_1 x_2$ $\Rightarrow$ b(v, v) = $x ^2$ and b(v, v) is never negative. However, b id degenerate because its matrix in the standard basis of $\mathbb(R)^2$ is b = 
\begin{pmatrix}
1 & 0\\
0 & 0
\end{pmatrix}
which is not invertible.
\end{exmp}
Let g be an inner product on V. \\
Two vectors u, v$\in$V are said to be
g-orthogonal if g(u, v) = 0. A null or isotropic vector of g is a vector that
g-orthogonal and null or
isotropic vectors
is g-orthogonal to itself. If g is definite, then the only null vector is the zero
vector. \\\\The converse is also true, as the following proposition shows

\begin{prop}
If g is not definite, then there exists a nonzero
isotropic vector.
\end{prop}
\begin{proof}
That g is not positive definite implies that there exists a nonzero vector v$\in$V such that g(v, v) $\leq$ 0. Similarly, that g is not negative definite
implies that there exists a nonzero vector w $\in$ V such that g(w, w) $\geq$ 0. Construct the vector u = $\alpha$v + (1 - $\alpha$)w and note that g(u,u) is a continuous function of $\alpha$. For $\alpha$ = 0 this function has the value g(w,w) $\geq$ 0, and for $\alpha$ = 1 it has the value g(v, v) $\leq$ 0. Thus, there must be some $\alpha$ for which g(u, u) = 0.
\end{proof}
\begin{defn}
Let g be an inner product on V and y a non-null (nonisotropic) vector in V. The $\textbf{projection}$ $x_y$ of x along y is given by 
\begin{center}
$x_y = \dfrac{g(x,y)}{g(y,y)} y$
\end{center}
\end{defn}
\begin{defn}
Let g be an inner product on V and y a non-null (nonisotropic) vector in V. The $\textbf{reflection}$ $x_{r,y}$ of x in a plane perpendicular to y are given by 
\begin{center}
$x_{r,y} = x - 2 \dfrac{g(x,y)}{g(y,y)} y$
\end{center}
\end{defn}

\subsection{Subscpaces}
Let V be a vector space with inner product g. Let W be a subspace of V.\\\\
Let W$^\perp$ be all vectors in V which are g-orthogonal to all vectors in W.
Ordinarily, we would call W$^\perp$ the orthogonal complement of W, but if g
is not definite, we can’t.
\begin{lem}
Let W be a subspace of a finite-dimensional inner product
space V. Then\\\\
1. dim W + dim W$^\perp$ = dim V\\
2. (W$^\perp$)$^\perp$ = W.
\end{lem}

\begin{proof}
1. Let \{e$_i$\}$^m_{i=1}$  be a basis of W with the dual basis \{$\epsilon$$_i$\}$^m_{i=1}$. Consider the linear operator $g_W$: V $\to$ W$^*$ given by
\begin{center}
$g_W(v) = \sum_{i=1}^{m} g(v, e_i)\epsilon^i$
\end{center}
 Using the dimension theorem ($\ref{dimthm}$) we can write
 \begin{center}
dim W$^*$ + dim ker g$_W$ = dim V.
\end{center}
Since dim W$^*$ = dim W, all that is left to show is that dim ker$g_W$ = dim W$^\perp$. We show more than that; we prove that ker$g_W = W^\perp$. In fact,\\
v $\in ker g_W$ \\
$\Leftrightarrow \sum^m_{i=1} g(v, e_i) \epsilon^i = 0$\\
$\Leftrightarrow g(v, e_i) = 0, i = 1, 2, ... , m. $\\
The last equality follows from the linear independence of  \{$\epsilon$$_i$\}$^m_{i=1}$, and it holds if and only if v $\in$ W$^\perp$.\\\\
2. If v $\in$ W, then v is orthogonal to all vectors in $W^\perp$, i.e., $v \in (W^\perp)^\perp$. Thus W $\subset$ (W$^\perp)^\perp$. Applying 1) to the subspace
W$^\perp$, we get dim $W^\perp$ + dim (W$^\perp)^\perp$ = dim V. Hence, dim W = dim (W$^\perp)^\perp$, and W = ( W$^\perp)^\perp$.
\end{proof}
A subspace W of an inner product space (V,g) is called $\textbf{nondegenerate}$ if g|$_W$ is nondegenerate. When g is definite, any subspace of V inherits
a definite inner product. Therefore, in this case every subspace is nondegenerate. However, when g is not definite, there will always be degenerate
subspaces. For example, if v is null, then the span of v is clearly degenerate.

\begin{prop}
A subspace W of an inner product space V is nondegenerate if and only if W$\oplus$W$^\perp$ = V.
\end{prop}

\begin{cor}
A subspace W of an inner product space is nondegenerate if and only if W$^\perp$ is nondegenerate.
\end{cor}
\begin{lem}
6 Let x and y be two vectors in V such that g(x, x) =
g(y, y) $\ne$ 0. Then there is a reflection R such that R(x) = $\pm y$.
\end{lem}
\begin{proof}
Because of the relation
\begin{center}
g(x + y, x + y) + g(x -y, x - y) = 4g(x, x) $\ne$ 0,
\end{center}
at least one of the terms on the left is nonzero. Assume that g(x -y, x - y) $\ne$, and let z = x - y. Then the reflection operator
\begin{center}
R$_z$ = 1 - 2P$_z$ = 1 - 2z $\dfrac{1}{g(z, z)} \phi_z$
\end{center}
is such that 
\begin{center}
$R_z(x) = x - 2z\dfrac{g(z,x)}{g(z,z)} = x - 2(x-y)\dfrac{g(x - y, x)}{g(x - y, x -y)}$\\
=$x - 2(x-y)\dfrac{g(x,x) - g(x, y)}{2g(x,x) - 2g(x,y)} = y.$
\end{center}
If g(x + y, x + y) $\neq$ 0, then let z = x + y. The reflection operator $R_z$, when acting on x, yields
\begin{center}
$R_z(x) = x - 2z\dfrac{g(z,x)}{g(z,z)} = x - 2(x + y)\dfrac{g(x+y,x)}{g(x+y, x+y)}$\\
= x - 2(x+y)$\dfrac{g(x,x) + g(x,y)}{2g(x,x) + 2g(x,y)} = -y.$
\end{center}
and the proof of the lemma is complete. 
\end{proof}
\begin{thm}
Let V be an N-dimensional inner product space.
Then any isometry T of V is the product of at most N + 1 reflections.
\end{thm}
\subsubsection{Orthonormal Basis}
Whenever there is an inner product on a vector space, there is the possibility
of orthogonal basis vectors. Since, in general, g(v, v) is allowed to be negative or zero, we have to redefine what we mean by a vector of norm 1. If
g(v, v) $\ne$ 0, we define the norm of v as ||v|| = |$\sqrt{g(v,v)}|. A unit vector, or
a vector of norm 1 obtained from v is simply v /||v||.
\begin{thm}
An inner product space V has an orthonormal basis.
\end{thm}
% TODO add sympletic group

% TODO add  inner product revisited



\newpage
\begin{thebibliography}{9}

\bibitem{1}  Hassani, S,

  \emph{Mathematical Physics: A Modern Introduction to Its Foundations}  2nd ed. Springer, 2013



\end{thebibliography}
\end{document}

